\documentclass{scrartcl}

\usepackage{fixltx2e}

% Step environment
% <https://tex.stackexchange.com/a/12943/13262>
\usepackage{amsthm}
\newtheorem*{remark}{Remark}
%
\newtheoremstyle{named}{}{}{\itshape}{}{\bfseries}{.}{.5em}{\thmnote{#1 }#3}
\theoremstyle{named}
\newtheorem*{step}{Step}

\usepackage{microtype}
\usepackage{amsmath}
\usepackage{mathtools}
\usepackage{booktabs}
\usepackage{tabularx}

\usepackage{pgfplots}
\pgfplotsset{compat=newest}

\usepackage{siunitx}

\newcommand\mytitle{How to assess color space quality}
\newcommand\myauthor{Nico Schlömer}

\usepackage[
  pdfencoding=unicode,
  ]{hyperref}
\hypersetup{
  pdfauthor={\myauthor},
  pdftitle={\mytitle}
}

% <https://tex.stackexchange.com/a/43009/13262>
\DeclarePairedDelimiter\abs{\lvert}{\rvert}%

\usepackage[T1]{fontenc}
\usepackage{newtxtext}
\usepackage{newtxmath}

% degree symbol
\usepackage{gensymb}

% % <https://tex.stackexchange.com/a/413899/13262>
% \usepackage{etoolbox}
% \makeatletter
% \long\def\etb@listitem#1#2{%
%   \expandafter\ifblank\expandafter{\@gobble#2}
%     {}
%     {\expandafter\etb@listitem@i
%      \expandafter{\@secondoftwo#2}{#1}}}
% \long\def\etb@listitem@i#1#2{#2{#1}}
% \makeatother

% Okay. Don't use biblatex/biber for now. There are breaking changes in every
% revision, and we'd have to stick to the exact version that arxiv.org has,
% otherwise it's error messages like
% ```
% Package biblatex Warning: File 'main.bbl' is wrong format version
% - expected 2.8.
% ```
% \usepackage[sorting=none]{biblatex}
% \bibliography{bib}

\usepackage{amsmath}
\DeclareMathOperator{\sign}{sign}
\DeclareMathOperator*{\argmin}{arg\,min}

% \usepackage{amsfonts}
\newcommand\R{\ensuremath{\mathbb{R}}}

\title{\mytitle\footnote{The LaTeX sources of this article are on
\url{https://github.com/nschloe/colorio}}}
\author{\myauthor}

\begin{document}

\maketitle
\begin{abstract}
  todo
\end{abstract}

\section{Introduction}

Every year, articles on new color spaces are created, claiming that they comply to
certain experimental data better than other color spaces. In many cases, numbers are
provided in form of a table showing that the new color space is indeed better.
Unfortunately, it is hardly ever explained how those numbers are computed.

This article describes in detail how to assess the experimental compliance of color
space with experimental data. The article focuses on the two most used types of
experimental data: Hue constancy \cite{ebner,xiao,hung} and color difference ellipses
\cite{macadam,luo}.

Almost every color space is defined by a transformation $T$ that maps the CIE-1931-XYZ
coordinates into a three-dimansional new coordinate space (e.g., LAB). The
transformation is usually continuously differentiable and bijective. When talking about
a colorspace, one almost always means the transformation $T$ and its inverse.


% There are numerous experiments~\cite{ebner,xiao,hung,macadam,macadam,luo} which try to
% gauge perceptual distances between colors or to determine which colors of different
% luminosity are perceived as the same chroma.
%
% These experimental data have been used in the past to approximate a perceptually uniform
% colorspace, i.e., a color space in which the Euclidean distance represents the perceived
% distance, and in which colors of same chroma all sit in one line. (These two goals are
% mutually inclusive.) So far, the general approach was to assume that the transformation
% from XYZ space takes a certain mathematical form with a number of free parameters, e.g.,
% the parameters $e$, $\alpha_{i,j}$, and $\omega_{i,j}$.
% \begin{equation}\label{eq:safdar}
%   \begin{split}
%     \begin{bmatrix}
%       L\\M\\S
%     \end{bmatrix}
%     =
%     \begin{bmatrix}
%       \alpha_{1,1} & \alpha_{1,2} & 1 - \alpha_{1,1} - \alpha_{1,2}\\
%       \alpha_{2,1} & \alpha_{2,2} & 1 - \alpha_{2,1} - \alpha_{2,2}\\
%       \alpha_{3,1} & \alpha_{3,2} & 1 - \alpha_{3,1} - \alpha_{3,2}
%     \end{bmatrix}
%     \begin{bmatrix}
%       X_{D65}\\Y_{D65}\\Z_{D65}
%     \end{bmatrix}\\
%     \{L',M',S'\} = \left(\frac{c_1 + c_2\left(\frac{\{L,M,S\}}{10000}\right)^n}{1 + c_3\left(\frac{\{L,M,S\}}{10000}\right)^n}\right)^{pe}\\
%     \begin{bmatrix}
%       I_z\\a_z\\b_z
%     \end{bmatrix}
%     =
%     \begin{bmatrix}
%       \omega_{1,1} & \omega_{1,2} & 1 - \omega_{1,1} - \omega_{1,2}\\
%       \omega_{2,1} & \omega_{2,2} &   - \omega_{2,1} - \omega_{2,2}\\
%       \omega_{3,1} & \omega_{3,2} &   - \omega_{3,1} - \omega_{3,2}
%     \end{bmatrix}
%     \begin{bmatrix}
%       L'\\M'\\S'
%     \end{bmatrix}\\
%   \end{split}
% \end{equation}
% in~\cite{safdar}. Then, an optimization algorithm was applied to retrieve those
% parameters which best match the given experimental data. The resulting color space has
% then been declared ``optimal'', which held true until a new article with a different
% assumption was published which, almost by chance, achieved even better accordance with
% the data.
%
% The assumption that the transformation of XYZ to the perceptually uniform color space
% takes a particular form is of course of practical nature: A low-dimensional parameter
% space is easy to search. However, to put it mildly, it is quite optimistic to assume
% that the visual system of the brain (see figure~\ref{fig:monkey}) does a transformation that
% can be expressed in terms of some linear transformations and elementary mathematical
% functions.
%
% TODO why doesn't polynomial approximation, pade not work? runge!
%
% \begin{figure}
%   \centering
%   \includegraphics[width=0.3\textwidth]{images/monkey.png}
%   \caption{Wiring diagram of the visual system of the macaque monkey, reproduced
%   from~\cite{felleman}.}
%   \label{fig:monkey}
% \end{figure}
%
% This article describes a much more general approach and succeeds in finding a color
% space that is far more perceptually uniform that everything that has been found so far.
% Even more, there can be no color space matching the given experimental data even better.
%
% \section{Optimization problem}
%
% There are many ideas that generalize the few-parameter approaches like~\ref{eq:safdar}.
% What comes to mind are polynomial approximations
% \begin{equation}\label{eq:poly}
%   p_{\alpha}(x, y) = \sum_{i+j\le n} \alpha_{i,j} x^i y^j
% \end{equation}
% where optimization happens over the coefficients $\alpha_{i,j}$, or even fractional
% polynomials,
% \[
%   r_{\alpha, \beta}(x, y) = \frac{p_\alpha(x,y)}{q_\beta(x, y)}
% \]
% where both numerator and denominator are of the form~\ref{eq:poly} (Padé approximant).
% Both of the approaches offer the advantage that -- given an infinite source of
% experimental data -- the actual transformation $t$ can be approximated arbitrarily well
% with increased polynomial degrees. Unfortunately, polynomial approximations suffer from
% Runge's phenomenon, meaning that naively chosen reference points for the optimization
% can lead solutions which approximate $t$ well at those points, but very badly everywhere
% else. See section~\ref{sec:polyfail}.
%
% This article takes a more robust approach. The key idea is to divide the domain into
% many small triangles (see figure~\ref{fig:triangles}) and to allow each of the nodes to
% move around more or less freely such that the resulting shape matches the experimental
% data well. All continuous transformations can be approximated by this approach so it is
% reasonable to assume that we do not restrict ourselves too much here.
%
% The mathematical concept
% \[
%   F(x, y) = \begin{bmatrix}a(x,y)\\b(x,y)\end{bmatrix}
% \]
% where both $a:\Omega\to\R^2$ and $b:\Omega\to\R^2$ are piecewise linear functions on the
% triangles.
%
% Each of the two
% \[
% fl
% \]
% \[
%   \begin{split}
%   F(a_x, a_y) &=\\
%   &F_{\text{Hung--Berns}}(a_x, a_y) +
%   F_{\text{Ebner--Fairchild}}(a_x, a_y) +
%   F_{\text{Xiao}}(a_x, a_y) +\\
%   &F_{\text{MacAdam}}(a_x, a_y) +
%   F_{\text{Luo--Rigg}}(a_x, a_y) +\\
%   &F_{\Delta}(a_x, a_y)
%   \end{split}
% \]
%
% % This is achieved by dividing the xy-triangle into many smaller triangles, i.e., nodes
% % and cells, and allowing each of the nodes to move around freely to match the
% % experimental data. This is the most general approach possible; all smooth
% % transformations can be expressed in this way.
%
% TODO all cost functions must be invariant to
% \begin{itemize}
%   \item rotation,
%   \item scaling, and
%   \item translation.
% \end{itemize}


\subsection{Cost functional for hue constancy data}

There are multiple experiments which data about which colors are perceived to be of
equal hue~\cite{ebner,xiao,hung} (see figure~\ref{}). Color spaces are considered good
if the transformation maps points of equal perceived hue onto a straight line.

What is a good measure of how well points sit on a straight line?
A common idea is to find the straight line that mimizes the sum of squared distances to
all points. This general approach is usually referred to as \emph{total least squares
(TLS)}.
The line is typically given implicitly by $\alpha_1 x + \alpha_2 y
= 0$ with $\alpha_1,\alpha_2\in\R$, $\|\alpha\|_2^2 = \alpha_1^2 + \alpha_2^2 = 1$.
To make sure the functional is scale-invariant and the line goes indeed through the
origin, translate it by the white point $w$ of the color space first and scale the
translated point cloud by its norm average.
In short:
\begin{equation}\label{eq:s}
s_2 \coloneqq
\min_{\|\alpha\|_2=1}
  \sum_{i=1}^n \sqrt{(\alpha_1 \tilde{x}_i + \alpha_2 \tilde{y}_i)^2}
\end{equation}
with the translated and scaled sample points
\[
  \tilde{x}_i = \frac{x_i-w_x}{\frac{1}{n}\sum_{j=1}^n (x_j-w_x)},\qquad
  \tilde{y}_i = \frac{y_i-w_y}{\frac{1}{n}\sum_{j=1}^n (y_j-w_y)}.
\]

The value of $s_2$ can be approximated with any appropriate optimization method. A
faster, more explicit, and more accurate representation however is retrieved as follows.
With the $n$-by-2 coordinate matrix
\[
  A \coloneqq \begin{pmatrix}
    \tilde{x}_1 & \tilde{y}_1\\
    \vdots & \vdots\\
    \tilde{x}_n & \tilde{y}_n
  \end{pmatrix},
\]
\eqref{eq:s} can be written as
\[
  s_2
  = \min_{\|\alpha\|_2=1} \sqrt{(A \alpha)^T (A \alpha)}
  = \min_{\|\alpha\|_2=1} \sqrt{\alpha^T A^T A \alpha}
\]
This makes clear that $s_2$ is exactly the square root of smaller of two eigenvalues of
$A^TA$ or equivalently the smaller of the two singular values of $A$,
\[
s_2
= \sqrt{\lambda_{\min}(A^T A)}
= \sigma_{\min}(A).
\]
The value is given explicitly by
\begin{equation}\label{eq:s2}
  s_2 = \sqrt{
    \frac{1}{2} \left(
      \tilde{x}^T\tilde{x}
      + \tilde{y}^T\tilde{y}
      - \sqrt{(\tilde{x}^T\tilde{x} - \tilde{y}^T\tilde{y})^2 + 4 (\tilde{x}^T\tilde{y})^2}
      \right)
    }.
\end{equation}


\begin{remark}
  The representation \eqref{eq:s2} is suitable for optimization purposes. Since a value
  $\sqrt{t}$ is small if and only if $t$ is small, one would in the interest of
  simplicity disregard the outer square root.
\end{remark}

\begin{remark}
Besides \eqref{eq:s}, there are other meaningful ways in which the distance of a point
cloud to a line can be defined. Instead of summing the squared distances, one could take
the $p$-norm,
\[
s_p
  \coloneqq \min_{\|\alpha\|_2=1} \|\alpha_1 \tilde{x} + \alpha_2\tilde{y}\|_p
= \min_{\|\alpha\|_2=1}
  \left(\sum_{i=1}^n |\alpha_1 \tilde{x}_i + \alpha_2 \tilde{y}_i|^p\right)^{1/p}
\]
  with $1\le p \le \infty$. The corner cases $p=1$ and $p=\infty$ (for which
  $s_{\infty}=\min_{\|\alpha\|_2=1} \max_i |\alpha_1 \tilde{x}_i + \alpha_2
  \tilde{y}_i|$) may be interesting. The value $s_2$ remains of prominent importance,
  though, because of its smooth dependence on the point set.
  Note also that, unlike often seen, in neither case the center of gravity
  $(\overline{x}, \overline{y})$ of the point set is of particular significance.
\end{remark}



\appendix

Let $(X_i, Y_i, Z_i)\in\R^3$ be a given set of points in XYZ space which, according to
some experiment, are of equal perceived hue. Consider the two non-lightness coordinates
of their image $(x_i, y_i) \coloneqq T(X_i, Y_i, Z_i)$ (see figure TODO). What is
measure of how well the points $(x_i, y_i)$ sit on a straight line? The general idea
here is to cast a line through ``the middle'' of the point cloud and sum up the
distances of all points to that line. There are multiple meaningful ways in which
``the middle'' and ``distance'' can be defined. Remember that in $\R^n$, the distance
between two points $a$ and $b$ can be defined by a \emph{norm}.



\subsection{The ellipse-circulator}

It might look surprising that the first experiments about color
perception by MacAdam~\cite{macadam} produced, of all possible shapes, rather regular
ellipses around their averages. This discovery, however, is less surprising if one
considers the fact that every continuously differentiable transformation from the
perceptually uniform space to xy will locally produce ellipses. That is because every
differentiable function $f:\R^n\to\R^n$ can be approximated locally by
\[
  f(\x) \approx f(\x_0) + J_{\x_0} (\x - \x_0).
\]
for any given point $\x_0$. The local behavior is thus determined by the $n\times n$
Jacobian matrix $J_{\x_0}$ at that point. In two dimensions, every linear transformation
maps circles into ellipses. Hence, the fact that we see ellipses hints toards there
being a continuously differentiable transformation from the perceptually uniform space
$S$ to $xy$.

This now also makes clear that the reverse transformation $t$ must \emph{undo} the
action of $J$, i.e., the transformation
\begin{equation}\label{eq:JJ}
  M_{\x_0} \coloneqq \tilde{J}_{t(\x_0)}^{-1} J_{\x_0}
\end{equation}
must map circles back into circles.
Demanding $M_{\x_0}$ to be the unit matrix is too
strict since we do not care about rotations. Rather consider its singular value
decomposition
\[
  M_{\x_0} =
  \begin{bmatrix}
    \cos \phi & \sin\phi\\
    -\sin \phi & \cos\phi
  \end{bmatrix}
  \begin{bmatrix}
    \sigma_1 & 0\\
    0 & \sigma_2
  \end{bmatrix}
  \begin{bmatrix}
    \cos \psi & \sin\psi\\
    -\sin \psi & \cos\psi
  \end{bmatrix}
\]
and note that the singular values $\sigma_1$, $\sigma_2$ are the magnitudes of the
semiaxes of the corresponding ellipses. Hence the cost functional should be small
if the values $\sigma_1$ and $\sigma_2$ are close to each other. Since there are
multiple ellipses involved and we would like all of them to map to circles of the same
size, all $\sigma$ across all ellipses should be of equal magnitude.

\begin{remark}
  Note that in equation~\ref{eq:JJ}, we could have switched the order of the matrices to
  \[
  \tilde{M}_{\x_0} \coloneqq J_{\x_0} \tilde{J}_{t(\x_0)}^{-1}
  \]
  The singular values of $\tilde{M}_{\x_0}$ do not equal those of $M_{\x_0}$.
  However, if the singular values of on the matrices are close to 1, so are those of the
  other.

  % s(t(x))
  %
  % J_st = J_s J_t ?
  % J_{s^-1} = (J_s)^{-1} ?
\end{remark}

Interestingly, the singular values of a $2\times 2$ matrix can be computed explicitly
via
\[
  \begin{split}
    &a,b = \frac{m_{1, 1} \pm m_{2, 2}}{2},\quad c,d = \frac{m_{2, 1} \pm m_{1, 2}}{2},\\
    &q = \sqrt{a^2 + d^2},\quad r = \sqrt{b^2 + c^2},\\
    &\sigma_{1/2} = q \pm r.
  \end{split}
\]
A reasonable cost functional would thus be
\[
  \begin{split}
  F &=
  \frac{1}{2} \sum_i {\left(\frac{q_i}{\overline{q}} - 1\right)}^2
  + \frac{1}{2} \sum_i {\left(\frac{r_i}{\overline{q}}\right)}^2\\
  % &= \frac{1}{2} \sum_i {\left(\frac{q_i}{\overline{q}} - 1\right)}^2
  % + \frac{1}{2} \sum_i {\left(\frac{b_i}{\overline{q}}\right)}^2
  % + \frac{1}{2} \sum_i {\left(\frac{c_i}{\overline{q}}\right)}^2
  \end{split}
\]
with $\overline{q} = \frac{1}{n} \sum_{j=1}^n q_j$. It is easy to verify that this
function fulfills the invariance conditions.

\section{}

For illustration, the efficiency of the cost functionals in sections~\ref{} and~\ref{}
will now be demonstrated with the few-parameter ansatz
\[
  t(x, y) = bla
\]




\subsection{The Laplacian}
\[
  |t(\Omega)|^{-1} \|\Delta t(x, y)\|
\]

The computation has been carried out with the help of FEniCS~\cite{fenics}.

\appendix

\section{Polynomial and Padé approximations and Runge's phenomenon}\label{polyfail}

See figure~\ref{} for the classical Runge example.


% \printbibliography{}
% \bibliography{pucs}{}
\bibliographystyle{plain}

\end{document}
